\documentclass[multi,crop=false,class=article]{standalone}
\onlyifstandalone{\usepackage{hyperref}
\usepackage{cleveref}
\usepackage[disable]{todonotes}
\presetkeys{todonotes}{inline, noline}{}
\usepackage{caption}
\usepackage{subcaption}
\usepackage{amsfonts}
\usepackage{theorem}
\usepackage{algorithm}
\usepackage{algpseudocode}
\theoremstyle{plain}
\theorembodyfont{\slshape}
\newtheorem{definition}{Definition}[section]
\usepackage{algorithm}
\usepackage{algpseudocode}
\usepackage{amsmath}
\usepackage{mathtools}
\usepackage{tikz}

\usetikzlibrary{automata,arrows}

%\DeclareCaptionType{algorithm}
\algdef{SE}[DOWHILE]{Do}{doWhile}{\algorithmicdo}[1]{\algorithmicwhile\ #1}%

% Math macros
\newcommand{\concat}{\cdot}
\newcommand{\bool}{\ensuremath{\mathbb{B}}}
\newcommand{\lang}{\ensuremath{\mathcal{L}}}
\newcommand{\dstar}[2]{\ensuremath{\delta^*(#1,#2)}}
\newcommand{\exec}[2]{\ensuremath{#1[#2]}}
\newcommand{\access}[2]{\ensuremath{#1^{-1}[#2]}}
\newcommand{\nero}{\ensuremath{\equiv}}
\newcommand{\eqclass}[2]{\ensuremath{[#1]_{#2}}}
\newcommand{\mq}{\ensuremath{\mathsf{member}}}
\newcommand{\eq}{\ensuremath{\mathsf{equiv}}}
\newcommand{\row}{\ensuremath{\mathsf{row}}}
\newcommand{\sift}{\ensuremath{\mathsf{sift}}}
%%% Local Variables:
%%% mode: latex
%%% TeX-master: t
%%% End:
}

\begin{document}
\section*{Introduction}
\label{sec:introduction}

Active state machine learning is a framework in which a learning algorithm (the
learner) is given access to a teacher to ask questions of different kind. The
goal of active state machine learning is to identify an unknown state machine.
It was first introduced by Dana Angluin in her paper \textit{Learning Regular
Sets from Queries and Counterexamples} in 1987\cite{Angluin1987}.

Active state machine learning tools are becoming increasingly prevalent in
today's world for many different applications. \todo{add references?} Examples
of such applications are protocol implementation validation and detection of
botnets. \todo{add references, more examples? (E.g. software testing)}

There are a plethora of papers and articles written about active state machine
learning. Many, however, use different terms, slight variations on algorithms
or data structures or are otherwise difficult to read. This leads to the
situation in which it is hard to get a proper overview of the subject.

The goal of this article, therefore, is to give a global overview of where the
technology of active state machine learning is currently at. While doing so, an
attempt is made to bring together all different terms used and to explain the
slight variations in algorithms. For this reason, this article should serve the
purpose of giving the reader a quick, concise but complete overview of not only
the current state of the art, but also the history of active state machine
learning and important advances made within this field of study over time.

To accomplish this, the first chapter explains the fundamental theory behind the
$L^{*}$ algorithm for learning regular sets. Chapter two discusses improvements
to the $L^{*}$ algorithm. Chapter three discusses the problems that arise when
applying the $L^{*}$ algorithm to real life problem domain(s) and the variatons
on the original algorithm that solve these problems.

The remaining chapters are concerned with applications of active state machine
learning. Chapter four discusses several tools that have been created to apply
active state machine learning. Lastly, important real world applications are
highlighted in chapter five.

\end{document}

%%% Local Variables:
%%% mode: latex
%%% TeX-master: "main"
%%% End:
