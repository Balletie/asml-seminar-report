\documentclass[11pt]{article}
\usepackage{hyperref}
\usepackage{sectsty}
\usepackage{todonotes}

\sectionfont{\large}

\title{\textbf{Short summary \& index of papers}}
\author{}

\date{\today}
\begin{document}

\maketitle

\section{\cite{Angluin87} Learning Regular Sets from Queries and Counterexamples}
Original paper by Dana Angluin.
Describes the $L^*$ algorithm for learning DFA's from regular sets.
Drawbacks:
\begin{itemize}
  \item applicable only to regular languages
  \item not very efficient, various improvements have been developed
  \item not very readable
  \item clearer and more modern summary
        \href{https://www.cs.bgu.ac.il/~beimel/Courses/Learning/}{here} and
        \href{https://www.cs.bgu.ac.il/~beimel/Courses/Learning/lect3.ps}{here}.
\end{itemize}

\section{\cite{Howar14} Tutorial: Automata Learning in Practice}
Lists and summarizes improvements that can be made to the $L^*$ learning algorithm.
Primarily focuses on removing sources of redundancy in membership queries.
\begin{itemize}
  \item TTT \cite{Isberner14b}, based on discrimination trees\cite{Kearns94}
  \item Observation pack algorithm
\end{itemize}

\section{\cite{deRuiter15} Protocol State Fuzzing of TLS Implementations}
Details learning a \textit{state machine model} for a given TLS implementation.
Knowledge gap between \cite{Angluin87} and this paper is quite large.
Required knowledge:
\begin{itemize}
  \item \cite{Angluin87} Basics of $L^*$ automata learning.
  \item \cite{Steffen12} Mapping input / output to DFAs.
\end{itemize}

\section{An Abstract Framework for Counterexample Analysis in Active Automata Learning\cite{Isberner14a}}
Presents a way for reducing counterexample analysis strategies to one common
framework. They also apply the framework to the Kearns and Vazarani's
discrimination tree algorithm\cite{Kearns94}. In so doing they also compare that
algorithm to the Observation Pack algorithm, which also uses a discrimination
tree but combines it with Rivest \& Schapire's counter example
analysis\cite{Rivest93}.

\section{\cite{Aarts13} Improving active Mealy machine learning for protocol conformance testing}
\section{\cite{Cho10} Inference and Analysis of Formal Models of Botnet Command and Control Protocols}
\section{\cite{Bauer14} Analyzing Program Behavior Through Active Automata Learning}
\section{Inference of Finite Automata Using Homing Sequences\cite{Rivest93}}
This paper gave two contributions. The first contribution is a way to learn DFAs
in an environment where a reset is not available when performing a membership
query.

The second one is a new method of counterexample analysis, also in case a reset
is available. It improves the upper bound of Angluin's counterexample analysis
from $\mathcal{O}(kn^2m)$ to
$\mathcal{O}(kn^2 + n \log(m))$\cite{Angluin87,Rivest93}. This is also an
improvement upon Kearn's and Vazirani's algorithm, where the counterexample
analysis had an upper bound of
$\mathcal{O}(kn^2 + nm)$\cite{Kearns94,Isberner15}.
\section{\cite{Steffen12} Active Automata Learning: From DFAs to Interface Programs and Beyond}

\bibliography{references}{}
\bibliographystyle{plain}
\end{document}
%%% Local Variables:
%%% mode: latex
%%% TeX-master: t
%%% End:
