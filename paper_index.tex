\documentclass[11pt]{article}
\usepackage{hyperref}
\usepackage{sectsty}
%\usepackage{cite}

\sectionfont{\large}

\title{\textbf{Short summary \& index of papers}}
\author{}

\date{\today}
\begin{document}

\maketitle

\section{\cite{Angluin87} Learning regular sets from queries and counterexamples}
Original paper by Dana Angluin.
Describes the $L^*$ algorithm for learning DFA's from regular sets.
Drawbacks:
\begin{itemize}
  \item applicable only to regular languages
  \item not very efficient, various improvements have been developed
  \item not very readable
  \item clearer and more modern summary
        \href{https://www.cs.bgu.ac.il/~beimel/Courses/Learning/}{here} and
        \href{https://www.cs.bgu.ac.il/~beimel/Courses/Learning/lect3.ps}{here}.
\end{itemize}

\section{\cite{deRuiter15} }



\bibliography{references}{}
\bibliographystyle{plain}
\end{document}