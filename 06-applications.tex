\documentclass[multi,crop=false,class=article]{standalone}

\begin{document}
\section{Applications}
\label{sec:applications}

\subsection{Tools}
\label{ssec:tools}

In order to facilitate and promote active automata learning in a practical 
setting, tools have been created. This section will give a brief overview of 
some tools that uses active automata learning. The tools Learnlib (section 
\ref{sssec:learnlib}) and Libalf (section \ref{sssec:libalf}) are tools that 
provides various active automata learning algorithms and optimizations whilst 
the tool Tomte(section \ref{sssec:tomte}) makes automatic abstractions for 
active automata learning. This selection is based on availability and 
researches that have used those tools \footnote{These include 
RALT\cite{Shahbaz2014} and AIDE \cite{Cicala2016}}. Also, tools that are used 
for automata moddeling and verification are also not included.

\subsubsection{Learnlib}
\label{sssec:learnlib}

Learnlib  \footnote{In active development and available from 
\url{https://github.com/LearnLib/learnlib} } is a library that implements 
various active 
learning algorithms as well as different configurations for learning 
automata. It has been in development since 2009 \cite{Raffelt2009} and as of 
2015, there has been a total overhaul of the tool \cite{Isberner2015}. To avoid 
confusion, the old version was renamed to JLearn. 

\todo{Image of architecture of Learnlib }
The current version consists out of two parts: Automatalib and Learnlib.

\todo{Image of architecture of Automatalib }

\paragraph{Automatalib} An independent library that contains abstract 
automata representations, automata data structures and algorithms. The abstract 
automata representations make the library flexible because all data structures
and algorithms depend on those representations. This makes it easy to add 
third party implementations of automata like the BRICS library 
\cite{Alur2005}. Automatalib included minimalization 
algorithms and equivalence testing algorithms based on Hopcroft and Karp's 
near-linear equivalence algorithm \cite{hopcroft1971} or the W-Method 
(See section X) %Need a reference to the W-Method  
for black-box testing. 

\paragraph{Learnlib} A library that provides learning algorithms and 
infrastructure for automata learning. The learning algorithms consist of a 
"base algorithm" whereby the counterexample analysis can be exchanged with 
other methods. All the different base algorithms with the examples of variants 
that are officially supported are listed below:

\todo{Should I give a really short explanation of methods that are not 
described in the theory?}
\begin{itemize}
	\item L* (base) (Explained in section %\ref{sec:fundamental-theory})
	\begin{itemize}
		\item Maler \& Pnueli's \cite{Maler1995}
		\item Rivest \& Schapnire's (Summarized in 
		%\ref{sec:rivest-schap-count)) 
		\item Shahbaz's \cite{Shahbaz2009}
		\item Suffix1by1 \cite{irfan2010}
	\end{itemize}
	\item Obversation Pack (base) \cite{howar2012a}
	\item Kearns \& Vazirani's (base) (Summarized in 
	%\ref{sec:classification-trees)
	\item DHC (base) \cite{Merten2012}
	\item TTT (base) %\todo{Reference to theory does not exists, yet}
	\item NL* \cite{Bollig2010}
\end{itemize}

All the algorithms come with both DFA and Mealy versions, expect for DHC and NL*.

For finding the counterexamples, Learnlib uses Automatalib as well as other 
methods like randomized tests. More methods can be found in \cite{Isberner2015}.

Learnlib also offers filters for reducing the amount of queries such as 
elimination of duplicate queries. %Is hier een theoriestuk over?
It also contains a parallellization component that can speed up the process by 
using multiple teachers and parallel execution of membership 
queries\cite{Henrix15}\cite{Howar2012}. 

\subsubsection{Libalf}
\label{sssec:libalf}
Libalf\footnote{Not in active development since 2011 but still available 
from \url{http://libalf.informatik.rwth-aachen.de}} is a library for learning 
and manipulating formal languages. It has both active and passive learning 
algorithms. For this paper, only the active part is covered.

\todo{Image of architecture of libalf }
The library consists of a core library as well as two additional libraries, 
liblangen(random regular language generator) and AMoRE++(automata library).

The core consists of the learning algorithms and the knowledgebase. The latter 
stores language information and collects the different queries that are used 
for that language. This storage makes it possible to switch learning algorithms 
or to use multiple learning algorithms during the learning process. The active 
learning algorithms that it offers are listed below.

\begin{itemize}
	\item L* (Explained in section %\ref{sec:fundamental-theory})
	\item Rivest \& Schapnire's (Summarized in 
	%\ref{sec:rivest-schap-count))
	\item NL* \cite{Bollig2010}
	\item Kearns \& Vazirani's (Summarized in 
	%\ref{sec:classification-trees)
\end{itemize}

Besides the algorithms and the knowledgebase, it has filters for reducing the 
amount of queries asked to the teacher. These filters use domain specific 
knowledge. %Is hier theorie over?
Also, it provides methods that uses domain specific equivalence relations 
supplied by the user to determine equivalence classes(data that is used by 
equivalence queries). This does reduce the amount of memory needed.

One major difference between Learnlib and Libalf is that Libalf does not 
include any methods that could be used for making a teacher. %Moet hier nog een 
%conclusie aan gebonden worden?

More information on Libalf is published in \cite{Bollig2010}.

\subsubsection{Tomte}
\label{sssec:tomte}
Tomte \footnote{In active development and available from 
\url{http://tomte.cs.ru.nl/Tomte-0-4}} is a tool that automatically makes 
abstractions for automata learning. Essentially, it a connector between the 
system under learning(SUL) and the learner. This makes using Learnlib and 
Libalf easier, since the user doesn't have to make the mapping.

\todo{Image of architecture of Tomte }

The Abstractor, Lookahead Oracle and the Determinizer together form Tomte. The 
other two parts are not part of Tomte but it comes with a supplied library 
(Learnlib) for making the learner.
The makers also have a tool \footnote{SUL Tool available from 
http://tomte.cs.ru.nl/Sut-0-4/Description} available for creating the SUL since 
they must be modeled after a register automata.
 
\paragraph{Determinizer}
This parts elimates the nondeterministic behaviour caused by the SUL. Since 
tools like Learnlib can only analyse deterministic behaviour, it needs to 
converted. The theory behind it, is explained in \cite{Aarts2015}.

\paragraph{Lookahead Oracle}
This oracle is used to annotate each output action of the SUL with values that 
have an impact on the future behaviour of the SUL. This makes it is possible to 
learn any deterministic register automaton. The theory and implementation of 
this oracle is found in \cite{Aarts2014} and \cite{tomte14}.

\paragraph{Abstractor}
The Abstractor is the component that creates the mapping between the SUL and 
the learner. The idea behind the Abstractor is to make an abstraction of the 
parameter values of the SUL but leaving the input/output symbols unchanged. It 
uses counterexample-guided abstraction refinement\cite{tomte14} for extension 
of the mapping. In order to make it scalable, this component also tries to 
reduce the length of the counterexample by removing loops and single 
transition \cite{Koopman2014}. The complete theory is found in \cite{tomte14}.


\todo{Make section about different researches that used those tools??}
%\cite{Aarts2014}\cite{Aarts2015}
%Biometric Passport \cite{Aarts2010}



In conclusion, from the comparisons it is clear that there is a lot of 
improvement in the creation of the tools in recent years by using the new 
methods that are found by either testing the tools on practical applications or 
new advances in active automata learning.


\end{document}

%%% Local Variables:
%%% mode: latex
%%% TeX-master: "main"
%%% End:
