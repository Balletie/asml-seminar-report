\documentclass[multi,crop=false,class=article]{standalone}
\onlyifstandalone{\usepackage{hyperref}
\usepackage{cleveref}
\usepackage[disable]{todonotes}
\presetkeys{todonotes}{inline, noline}{}
\usepackage{caption}
\usepackage{subcaption}
\usepackage{amsfonts}
\usepackage{theorem}
\usepackage{algorithm}
\usepackage{algpseudocode}
\theoremstyle{plain}
\theorembodyfont{\slshape}
\newtheorem{definition}{Definition}[section]
\usepackage{algorithm}
\usepackage{algpseudocode}
\usepackage{amsmath}
\usepackage{mathtools}
\usepackage{tikz}

\usetikzlibrary{automata,arrows}

%\DeclareCaptionType{algorithm}
\algdef{SE}[DOWHILE]{Do}{doWhile}{\algorithmicdo}[1]{\algorithmicwhile\ #1}%

% Math macros
\newcommand{\concat}{\cdot}
\newcommand{\bool}{\ensuremath{\mathbb{B}}}
\newcommand{\lang}{\ensuremath{\mathcal{L}}}
\newcommand{\dstar}[2]{\ensuremath{\delta^*(#1,#2)}}
\newcommand{\exec}[2]{\ensuremath{#1[#2]}}
\newcommand{\access}[2]{\ensuremath{#1^{-1}[#2]}}
\newcommand{\nero}{\ensuremath{\equiv}}
\newcommand{\eqclass}[2]{\ensuremath{[#1]_{#2}}}
\newcommand{\mq}{\ensuremath{\mathsf{member}}}
\newcommand{\eq}{\ensuremath{\mathsf{equiv}}}
\newcommand{\row}{\ensuremath{\mathsf{row}}}
\newcommand{\sift}{\ensuremath{\mathsf{sift}}}
%%% Local Variables:
%%% mode: latex
%%% TeX-master: t
%%% End:
}

\begin{document}
\section{Improvements}
\label{sec:improvements}
There are several improvements to be made with respect to the original $L^*$
algorithm described in \cref{sec:fundamental-theory}, which will be discussed in
this chapter. In discussing the improvements, we make two kinds of comparisons
between the algorithms. The first comparison is in terms of the amount of
\textit{queries} (the \textit{query complexity}), which gives an indication of
the running time of the algorithm.

In the second comparison, we show the improvements in terms of \textit{space
  complexity} between algorithms. We discuss the TTT-algorithm, which according
to Isberner~et~al. exhibits an optimal space complexity\cite{Isberner14b}.

The first improvement which will be discussed is the \textit{classification
  tree}, which forms the basis for multiple subsequent improvements (including
the TTT-algorithm).

\subsection{Classification Trees}
\label{sec:classification-trees}
Introduced by Kearns and Vazirani\cite{Kearns94}, classification trees (or
\textit{discrimination trees}) are meant as a replacement for the observation
table (see \cref{sec:data-structure}). In other words, it is a different data
structure for storing the sets $S$ and $E$ of access strings and distinguishing
extensions. The main characteristic of a classification tree is that it is
\textit{redundancy-free}, which will be explained in this section.

Say that we have a classification tree for some target automaton $M$. The labels
of the internal nodes of the tree are distinguishing extensions from the set
$E$, and the labels of leaf nodes are access strings from $S$. The tree's
structure is such that for any internal node $e \in E$, its left subtree
contains all $s \in S$ such that $s \concat e$ is rejected by $M$, and its right
subtree contains those $s$ such that $s \concat e$ is accepted by $M$.

From the way the tree is constructed, any pair of access strings $v,w \in S$ are
distinguished by their \textit{lowest common ancestor} in the tree. Thus, $v$
and $w$ are distinguished by exactly \textit{one} distinguishing extension. In
other words, any pair of access strings in the classification tree are
Nirode-inequivalent~\todo{reference subsection chpt 2}, and thus each access
string uniquely identifies a distinct equivalence class of the target automaton
$M$. When we construct a hypothesis, each equivalence class (and thus each
access string), corresponds to a state in our hypothesis automaton $\hat M$.

Howar~et~al. say that a discrimination tree is \textit{redundancy-free} because
each state in $\hat M$ is distinguished by one distinguishing extension, as
opposed to observation tables where states are distinguished by a fixed amount
of distinguishing extensions\cite{Howar14}.

\subsubsection{Complexity analysis}
\label{sec:complexity-analysis}
With their introduction of the \textit{classification tree}, Kearns and Vazirani
also adapted the original $L^*$ learning algorithm to use the tree
structure\cite{Kearns94}. The precise working of the adapted algorithm will not
be given, but we do give an overview of the impact on the query complexity and
the space complexity.

\begin{description}
\item[Query Complexity] The amount of \textit{membership queries} for hypothesis
  construction is bounded by $\mathcal{O}(kn^2)$ membership queries in the
  worst-case (i.e. a degenerated tree)\cite{Howar14,Kearns94,Isberner14b}, where
  $k = |\Sigma |$ and $n$ is the size of the target automaton $M$. For the
  counterexample analysis the amount of queries is bounded by $\mathcal{O}(nm)$,
  where $m$ is the size of the longest counterexample\cite{Kearns94}.

  Thus in total the query complexity is bounded by $\mathcal{O}(kn^2 + nm)$, an
  improvement over that of $L^*$.
\item[Space Complexity] In chapter 4 of his dissertation, Isberner analyzed the
  space complexity of discrimination-tree based learners as follows: a
  degenerated tree has $\mathcal{O}(n)$ amount of nodes, of which each inner
  node is labeled by a distinguishing extension of length
  $\mathcal{O}(m)$\cite{Isberner15}.  The space required for the hypothesis
  automaton is $\mathcal{O}(kn)$\cite{Isberner15}. Thus the algorithm requires
  space in $\mathcal{O}(kn + mn)$\cite{Isberner14b,Isberner15}\todo{Compare with
    L-star}.
\end{description}

\end{document}

%%% Local Variables:
%%% mode: latex
%%% TeX-master: "main"
%%% End: