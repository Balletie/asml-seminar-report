\documentclass[multi,crop=false,class=article]{standalone}

\begin{document}
\section*{Conclusion}
\label{sec:conclusion}
In this article, the history and development of active state machine learning
have been discussed. Chapter one explained the fundamental theory behind the
topic, providing a formal basis for the rest of the article. The notions of
Nerode-equivalence and equivalence classes were introduced. Furthermore, it was
explained what purpose active state machine learning serves. The main algorithm,
the one described by Angluin~\cite{Angluin1987}, was explained in detail. These
key concepts are fundamental to all further research and development performed
on the topic.


In chapter two, various improvements on the original research are discussed,
such as the classification tree as introduced by Kearns and Vazirani~\cite{Kearns1994}.
This different type of datastructure is, as opposed to the discrimination table
defined by Angluin, proven to be redundancy-free, resulting in more efficient
query and space complexities. Rivest and Schapire~\cite{Rivest1993} introduced
another significant improvement, namely counterexample analysis. Instead of
adding all prefixes from a counterexample to the set of distinguishing
extensions, only the one leading to an incorrect transition is added. When it
comes to query complexity, Rivest and Schapire achieve even more improvement
than Kearns and Vazirani. However, their demand for space is larger, albeit
still improving on $L^*$. A combination of these two improvements was introduced
by Howar~\cite{Howar2012a,Isberner2015a}: the Observation Pack algorithm.
Building even further on these improvements, Isberner~\cite{Isberner2014b} came
up with the TTT algorithm, focusing on keeping counterexamples and thus the
classification tree as concise as possible by removing the redundancy from
counterexamples given by the Teacher.


In chapter three, numerous variants arising from problems in applying
the theory in practice were discussed. It was explained that the W-method,
as defined by Chow~\cite{deRuiter2015,Chow1978}, can be used to approximate equivalence
queries with membership queries. Chow also showed that this is a decidable problem
when given an estimate of the maximum number of states of the SUL.
Since the number of membership queries required grows even more when also using
them for approximation of equivalence queries, algorithms benefit greatly from
knowledge of application-specific context.
Shahbaz and Groz~\cite{Shahbaz2009}
introduced the notion of Mealy machines combined with active state machine
learning, adding the support for I/O-dependent machines that $L^*$ lacks. For
this, the $L^*$ algorithm was modified to output and store strings instead of
either true or false. To eliminate the problem of being unable to reset the
machine between individual membership queries, Rivest and Schapire~\cite{Rivest1993}
proposed the use of homing sequences. When executed, these sequences lead to the
final state of the machine, simulating a full reset.


In chapter four, Learnlib, libalf and Tomte~\cite{Raffelt2009,Bollig2010,Tomte2014}
were discussed to illustrate how active state machine learning can be modeled in
practice. Learnlib provides the user with numerous learning algorithms to choose
from, including $L^*$ and many of its improved variants and supports both DFAs
and Mealy machines, depending on the chosen algorithm. Tomte is a tool that maps
the SUL to the Learner. It first solves nondeterminism by conversion and then
preprocesses the SUL behavior, before actually mapping the SUL to the Learner.


The final chapter, chapter
five, highlighted practical applications of state machine learning.
Protocol validation~\cite{deRuiter2015} was found to be applicable in reality by
adding protocol-specific improvements to the learning algorithms. In particular
knowledge about the system that reduces the execution time or even enables the
learning algorithm to be interrupted after a certain state had been reached.
\end{document}

%%% Local Variables:
%%% mode: latex
%%% TeX-master: "main"
%%% End:
