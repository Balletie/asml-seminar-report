\documentclass[multi,crop=false,class=article]{standalone}

\newcommand{\concat}{\cdot} 

\begin{document}
\section{Fundamental Theory}
\label{sec:fundamental-theory}

Angluin defines the $L^*$ algorithm for learning regular sets. The goal of the
algorithm is to discover a deterministic finite automaton (DFA) that corresponds
to the regular set.

The regular set is assumed to contain only \textit{words}, where each word
consists of the concatenation of zero or more items from a certain set of of
symbols called the input alphabet $A$.

The algorithm requires a so called \textit{minimally adequate teacher} or
\textit{oracle} which can answer two types of questions about the regular set.
The first is a membership query; it answers whether a given word is in the
regular set. The second is an equivalence query: given a DFA model, it answers
`yes' if the model exactly matches the regular set, or provides counterexample
(i.e. a word that is in the regular set but not in the DFA or vice versa) if it
does not. With these two questions, any regular set can be learned.

\subsection {Data structure} The algorithm keeps track of a set of \textit{state
candidates} $S$ and a set of distinguishing experiments  $E$. The answer to the
queries are stored in a two dimensional table called an observation table. The
columns of this table are headered by the items of $E$. The rows are split into
two parts. The rows of the upper and lower parts are headers by items from $S$
and $S \concat A$ respectively, where `$\concat$' is the concatenation operator.
Consider a cell at row $s$ and column $e$. The row represents the resulting
state when input $s$ is applied to the initial DFA state. The value at column
$e$ represents whether inputing $e$ from that state results in an accepting
state. Equivalently, this can be interpreted to mean whether inputing $s \concat
e$ from the initial state of the DFA results in an accepting state. Two rows in
the observation table are considered to represent the same state if they react
identically under the same input (i.e. if they have the same values under all
columns).

\subsection {The algorithm} The algorithm consists of two phases. The first
phase is repeated until a closed and consistent model is found.

A model is called inconsistent if and only if the observation table contains
distinct rows with identical values under $E$ (i.e. the rows appear to represent
the same state), but inputing some $a \in A$ results in rows with different
values in some $e \in E$. Thus if a model is found to be inconsistent, feeding
the input $a \concat e$ results in different states, and so $a \concat e$ is
discovered to be a distinguishing experiment. Therefore, $a \concat e$ is added
$E$, making the two rows in $S$ become distinct under the columns of $E$, thus
removing the inconsistency. New membership queries are performed in order to
fill in the blank cells in the table.

% dit moet nog even wat beter.. 
A model is considered to be closed if $S \concat A$ doesn't contain any states
that are not in $S$. If there are, the row is moved to $S$. The reason for this
is that if there is a row $x$ in $S \concat A$ that is not in $S$, then it is
unknown how $x$ reacts to $A$, so the current hypothesis of the state machine
does not describe what happens in state $x \concat A$. After moving $x$ to $S$,
the rows $x \concat A$ are added to the lower part of the observation table, and
membership queries are performed the fill in the cells. 
\\\\
In the second phase of the algorithm, the equivalence check is
performed. If the reply is `yes', then the algorithm stops. If instead a counter
example is replied, the counterexample and all of its prefixes are added to $S$,
after which $S \concat A$ is updated and the algorithm moves back to phase 1.

\subsection {Equivalence queries} In practice, no oracle is available to
precisely answer equivalence queries, because if the oracle knew the exact DFA,
the whole algorithm wouldn't be required. Instead, the answers to equivalence
queries are approximated by repeatedly running random membership queries. For
each of these membership queries, a random sequence of symbols from in input
alphabet is generated, and is then checked against both the proposed DFA and the
unknown regular set. If they do not match, the counterexample is found. This
process is repeated until the desired confidence that the statemachine is
correct is reached.

\subsection {Example run}

\end{document}

%%% Local Variables:
%%% mode: latex
%%% TeX-master: "main"
%%% End: