\documentclass[multi,crop=false,class=article]{standalone}
\onlyifstandalone{\usepackage{hyperref}
\usepackage{cleveref}
\usepackage[disable]{todonotes}
\presetkeys{todonotes}{inline, noline}{}
\usepackage{caption}
\usepackage{subcaption}
\usepackage{amsfonts}
\usepackage{theorem}
\usepackage{algorithm}
\usepackage{algpseudocode}
\theoremstyle{plain}
\theorembodyfont{\slshape}
\newtheorem{definition}{Definition}[section]
\usepackage{algorithm}
\usepackage{algpseudocode}
\usepackage{amsmath}
\usepackage{mathtools}
\usepackage{tikz}

\usetikzlibrary{automata,arrows}

%\DeclareCaptionType{algorithm}
\algdef{SE}[DOWHILE]{Do}{doWhile}{\algorithmicdo}[1]{\algorithmicwhile\ #1}%

% Math macros
\newcommand{\concat}{\cdot}
\newcommand{\bool}{\ensuremath{\mathbb{B}}}
\newcommand{\lang}{\ensuremath{\mathcal{L}}}
\newcommand{\dstar}[2]{\ensuremath{\delta^*(#1,#2)}}
\newcommand{\exec}[2]{\ensuremath{#1[#2]}}
\newcommand{\access}[2]{\ensuremath{#1^{-1}[#2]}}
\newcommand{\nero}{\ensuremath{\equiv}}
\newcommand{\eqclass}[2]{\ensuremath{[#1]_{#2}}}
\newcommand{\mq}{\ensuremath{\mathsf{member}}}
\newcommand{\eq}{\ensuremath{\mathsf{equiv}}}
\newcommand{\row}{\ensuremath{\mathsf{row}}}
\newcommand{\sift}{\ensuremath{\mathsf{sift}}}
%%% Local Variables:
%%% mode: latex
%%% TeX-master: t
%%% End:
}

\begin{document}
\section{Applications}
\label{sec:applications}

This chapter serves to highlight some interesting and important practical
applications of active state machine learning. The goal is not to discuss
individual applications, rather several ``domains'' of application have been
identified: protocol verification and  botnet identification.
Each domain will be introduced shortly and illustrated with a short example.

\subsection{Protocol Verification}
One commonly used application of active state machine learning is
protocol verification. This checks whether a protocol implementation follows
its formal description. The description is used to identify states and
allowed transitions. These states and transitions are then used to construct a
state machine that conforms to the protocol definition. It is this machine
against which equivalence queries are performed, to find out whether the
implementation logic is according to the specification. The input alphabet
is defined as an abstract set of messages that relate to the messages the
protocol specification defines. Furthermore, an output alphabet is defined,
containing a set of abstractions of the possible messages the protocol
implementation can output. Often, the output alphabet is concatenated with
some special strings that trigger a certain command, such as a reset command.
Later on in this section, an example of such an extension is given.

Sequences of input messages are constructed and passed to the SUL. If the
machine's output matches the expected output, the implementation is correct.
However, if the expected output does not match the actual output, there is
some bug in the SUL. The tests are performed as black box testing, resulting
in the need for approximation as explained in \cref{sec:variants}. One specific
example of protocol verification is given below. More examples of applications
can be found in~\cite{Aarts2013,Cho2010,Aarts2010}.

\subsubsection{TLS-protocol verification} For this specific protocol verification
application~\cite{deRuiter2015}, an improved version of the W-method as explained
in \cref{sec:chow} is used. The improvement comes forth from the fact that the
TLS-protocol requires the output to be ``Connection closed'' at all times once a
connection is closed. By making use of this and thus limiting the W-method to
halt whenever the connection is closed, a great reduction in the time needed to
perform the implementation analysis is achieved, especially since the W-method is
a very costly one.

An aspect that has proven to be difficult in practice, namely
that of resetting the state of the machine after each iteration, is for this
specific application fairly easy. De Ruiter et al. defined an additional
message, a ``reset''-message. When this message is used, a new client-server 
connection is set up after which a new iteration can start.

\subsection{Learning Botnets}
Related to the application of protocol verification is applying active state
machine learning to botnet Command and Control (CnC) protocols~\cite{Cho2010}.
An implementation of a botnet is very well comparable to that of a protocol
implementation. By learning the protocol of a botnet, knowledge about it
can be inferred, that might ultimately lead to exposing vulnerabilities
that can be used to protect against it or even bring it down.

\subsubsection{MegaD analysis}
Cho~et~al. analyzed the CnC protocol of the MegaD botnet.
For this, several improvements have been made to the $L^*$ algorithm.
They wrote a script that communicates with the botnet. It concretizes input
sequences to valid protocol messages and converts the response of the botnet
back to contain only elements defined in the output alphabet. Eventually, this
response is returned to the learning algorithm.

Since a botnet CnC protocol inherently operates over the internet, queries have
a high latency. Caching queries combined with parallel execution increased the execution speed
of the inference by approximately 4.85 times compared to inference using purely
$L^*$~\cite{Cho2010}. The main speed gain came from the ability to send
a unique sequence of input messages only once, leading to a great reduction in
the total latency suffered from sending messages over the network.

Finally, they used a recursive approach to be able to use predictions of
membership query results and to roll back to a previous state if a prediction
were to be found erroneous (identified by random sampling). These predictions
were mainly feasible because the inference process initially had to handle the
many self-loops of states. An example of such self-loop is when the SUL is
expecting a specific message, but receives another. An error is generated and
the client may try again. Since many different kinds of unexpected messages may
exist, the number of self-loops can also be numerous.

The common issue of resettability was solved by Cho~et~al. by initializing a
new session before sending a new (sequence of) message(s). A source of
non-determinism they encountered was the master server responding with a
continuous stream of INFO-messages, which theoretically would not have to end.
This was solved by setting a maximum threshold on the latency allowed before an
answer had to be arrived. If this threshold was reached, they simply reset the
session.

\end{document}

%%% Local Variables:
%%% mode: latex
%%% TeX-master: "main"
%%% End:
