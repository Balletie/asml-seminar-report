\documentclass[multi,crop=false,class=article]{standalone}

\begin{document}
\section{Variants}
\label{sec:variants}
In the first few years following the introduction of the L* algorithm in
\cite{Angluin87}, it was only a theoretical exploration of active learning.
The various improvements described in the previous chapter,
such as \todo{Cite some important improvements here},
made practical applications suddenly more feasible.

In applying L* or improvements thereof to practical applications,
there were still some practical limitations and restrictions that had to
be addressed (Steffen 2011, Ch. 6 \cite{Steffen11}).
Some of these issues can be summarized as follows:
\begin{itemize}
  \item Equivalence queries are generally undecidable
  \item L* can only interact with regular languages, not with real systems
  \item The amount of required membership queries can grow very fast
  \item L* has no way of dealing with parameter and value domains
  \item Membership queries might not be independent in practice.
        This might sometimes be solved by \textit{reset}ing the system.
        However, not all systems might support a \textit{reset}.
\end{itemize}

In order to address these limitations and restrictions,
several variants of the L* algorithm have been proposed,
each with the goal to solve such an issue.
In this chapter we will further elaborate on some of these issues
and the variants proposed to solve them.

\subsection{Equivalence queries are generally undecidable}
The target machine on which the learning algorithm is applied is generally
a black box, since we want to infer knowledge about some unknown system.
Therefore equivalence queries can only be answered by exhaustively testing
the inputs of the system. This problem has been shown to be undecidably hard.

In order to solve this issue model-based testing techniques have been used,
such as \textit{Chow's W-method} or the \textit{WP-method} \todo{Cite here}.
These methods rely on approximating \textit{EQUIV(M)} queries by using
\textit{MEMBER(w)} queries.

To further research in this area, the \textit{ZULU} challenge was introduced.
The \textit{ZULU} challenge asked participants to find a DFA corresponding
to a certain system as accurately as possible, while imposing a restriction
on the number of \textit{MEMBER(w)} queries and disallowing \textit{EQUIV(M)}
queries completely.
\todo{Possibly summarize some interesting result, like the winner.}

\subsection{L* can only interact with regular languages, not with real systems}
% @sander: take it away? :)

\subsection{The amount of required membership queries can grow very fast}
In a theoretical framework the learning algorithm doesn't have to account
for execution times of individual \textit{MEMBER(w)} queries.
In practice, such queries might take time or be expensive to execute.
Therefore algorithms benefit from reducing the amount of queries needed.
Some generic improvements in this regard have already been covered in
\todo{Reference previous chapter}.

However, when using for example \textit{Chow's method} for \textit{EQUIV(M)}
queries, the amount of membership queries needed grows exponentially
in the number of states \todo{Cite source for exponential complexity of Chow's}.
(...)

\subsection{Systems without a reset}

\end{document}

%%% Local Variables:
%%% mode: latex
%%% TeX-master: "main"
%%% End:
