\documentclass[multi,crop=false,class=article]{standalone}
\onlyifstandalone{\usepackage{cleveref}
\usepackage{todonotes}

\newcommand{\concat}{\cdot} 

%%% Local Variables:
%%% mode: latex
%%% TeX-master: t
%%% End:}

\begin{document}
\section{Improvements}
\label{sec:improvements}
There are several improvements to be made with respect to the algorithms
described in \cref{sec:variants,sec:fundamental-theory}, using different data
structures which decrease the space complexity.

\subsection{Classification Trees}
\label{sec:classification-trees}
Introduced by Kearns and Vazirani\cite{Kearns94}, classification trees (or
\textit{discrimination trees}) were meant as a replacement for the observation
table (see \cref{sec:data-structure}). In other words, it is a different data
structure for storing the sets $S$ and $E$ of access strings and distinguishing
extensions. The main characteristic of a classification tree is that it is
\textit{redundancy-free}, which will be explained in this section.

Say that we have a classification tree for some target automaton $M$. The labels
of the internal nodes of the tree are distinguishing extensions from the set
$E$, and the labels of leaf nodes are access strings from $S$. The tree's
structure is such that for any internal node $e \in E$, its left subtree
contains all $s \in S$ such that $s \concat e$ is rejected by $M$, and its right subtree
contains those $s$ such that $s \concat e$ is accepted by $M$.

From the way the tree is constructed, any pair of access strings $v,w \in S$ are
distinguished by their \textit{lowest common ancestor} in the tree. Thus, $v$
and $w$ are distinguished by exactly \textit{one} distinguishing extension. In
other words, any pair of access strings are Nirode-inequivalent~\todo{reference
  subsection chpt 2}, and thus each access string uniquely identifies a distinct
equivalence class of the target automaton $M$. When we construct a hypothesis,
each equivalence class (and thus each access string), corresponds to a state in
our hypothesis automaton $\hat M$.

Howar~et~al. note that this means that a discrimination tree is
\textit{redundancy-free}, as opposed to observation tables where states are
distinguished by a fixed amount of distinguishing extensions\todo{Verify
  this. What do Howar et al mean with ``fixed number of
  discriminators''?}\cite{Howar14}.
\end{document}

%%% Local Variables:
%%% mode: latex
%%% TeX-master: "main"
%%% End:
