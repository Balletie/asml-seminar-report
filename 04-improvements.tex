\documentclass[multi,crop=false,class=article]{standalone}
\onlyifstandalone{\usepackage{hyperref}
\usepackage{cleveref}
\usepackage[disable]{todonotes}
\presetkeys{todonotes}{inline, noline}{}
\usepackage{caption}
\usepackage{subcaption}
\usepackage{amsfonts}
\usepackage{theorem}
\usepackage{algorithm}
\usepackage{algpseudocode}
\theoremstyle{plain}
\theorembodyfont{\slshape}
\newtheorem{definition}{Definition}[section]
\usepackage{algorithm}
\usepackage{algpseudocode}
\usepackage{amsmath}
\usepackage{mathtools}
\usepackage{tikz}

\usetikzlibrary{automata,arrows}

%\DeclareCaptionType{algorithm}
\algdef{SE}[DOWHILE]{Do}{doWhile}{\algorithmicdo}[1]{\algorithmicwhile\ #1}%

% Math macros
\newcommand{\concat}{\cdot}
\newcommand{\bool}{\ensuremath{\mathbb{B}}}
\newcommand{\lang}{\ensuremath{\mathcal{L}}}
\newcommand{\dstar}[2]{\ensuremath{\delta^*(#1,#2)}}
\newcommand{\exec}[2]{\ensuremath{#1[#2]}}
\newcommand{\access}[2]{\ensuremath{#1^{-1}[#2]}}
\newcommand{\nero}{\ensuremath{\equiv}}
\newcommand{\eqclass}[2]{\ensuremath{[#1]_{#2}}}
\newcommand{\mq}{\ensuremath{\mathsf{member}}}
\newcommand{\eq}{\ensuremath{\mathsf{equiv}}}
\newcommand{\row}{\ensuremath{\mathsf{row}}}
\newcommand{\sift}{\ensuremath{\mathsf{sift}}}
%%% Local Variables:
%%% mode: latex
%%% TeX-master: t
%%% End:
}

\begin{document}
\section{Improvements}
\label{sec:improvements}
There are several improvements to be made with respect to the algorithms
described in \cref{sec:variants,sec:fundamental-theory}, using different data
structures which decrease the space complexity.

\subsection{Classification Trees}
\label{sec:classification-trees}
Introduced by Kearns and Vazirani\cite{Kearns94}, classification trees (or
\textit{discrimination trees}) were meant as a replacement for the observation
table\todo{reference subsection chpt 2}. In other words, it is a different data
structure for storing the sets $S$ and $E$ of access strings and distinguishing
extensions. The main characteristic of a classification tree is that it is
\textit{redundancy-free}, which will be explained in this section.

Say that we have a classification tree for some target automaton $M$. The labels
of the internal nodes of the tree are distinguishing extensions from the set
$E$, and the labels of leaf nodes are access strings from $S$. The tree's
structure is such that for any internal node $e \in E$, its left subtree
contains all $s \in S$ such that $s \concat e$ is rejected by $M$, and its right subtree
contains those $s$ such that $s \concat e$ is accepted by $M$.

From the way the tree is constructed, any pair of access strings $v,w \in S$ are
distinguished by their \textit{lowest common ancestor} in the tree. Thus, $v$
and $w$ are distinguished by exactly \textit{one} distinguishing
extension. Furthermore, it follows from the Myhill-Nerode
Theorem~\todo{reference subsection chpt 2} that the strings $v$ and $w$ are
Nirode-inequivalent, and thus each string represents a distinct equivalence
class\todo{``candidate-state''?}. Howar~et~al. note that from this it follows
that a discrimination tree is \textit{redundancy-free}, as opposed to
observation tables where states can be distinguished by multiple distinguishing
extensions\todo{Verify this. What do Howar et al mean with ``fixed number of
  discriminators''?}\cite{Howar14}.
\end{document}

%%% Local Variables:
%%% mode: latex
%%% TeX-master: "main"
%%% End:
