\documentclass[multi,crop=false,class=article]{standalone}

\begin{document}
\section{Tools}
\label{sec:tools}

In order to facilitate and promote active automata learning in a practical
setting, tools have been created. This section will give a brief overview of two
tools. First, the tool Learnlib (\ref{ssec:learnlib}) will be discussed. This
tool provide various active automata learning algorithms and optimizations. The
second tool is Tomte(\ref{ssec:tomte}), which automatically generates
abstractions for automatons in order to apply active automata learning on them.
This selection is based on the importance of the tool, availability and the
amount of researches that have used those tools.

\subsection{Learnlib}
\label{ssec:learnlib}

Learnlib \footnote{Supported through bug fixes and available from
\url{https://github.com/LearnLib/learnlib}} is a library that implements various
active learning algorithms as well as different configurations for learning
automata. It has been in development since 2009 \cite{Raffelt2009} and as of
2015, there has been a total overhaul of the tool \cite{Isberner2015}. To avoid
confusion, the old version was renamed to JLearn.

The current version of this tool consists out of two parts: Automatalib and
Learnlib.

\begin{figure}[!ht]
	\includegraphics[width=\textwidth]{Tool_images/automatalib_architecture.png}
	\caption{Architecture of Automatalib, source is from \cite{Isberner2015}}
	\label{fig:automatalib_arch}
\end{figure}

\paragraph{Automatalib} An independent library that contains abstract automata
representations, automata data structures and algorithms. The abstract automata
representations make the library flexible because all data structures and
algorithms depend on those representations. This makes it easy to add third
party implementations of automata like the BRICS library \cite{Alur2005}.
Automatalib includes minimalization algorithms and equivalence testing
algorithms based on Hopcroft and Karp's near-linear equivalence algorithm
\cite{hopcroft1971} or the W-Method (X) \todo{Need a reference to the W-Method}
for black-box testing.

\begin{figure}[!ht]
	\includegraphics[width=\textwidth]{Tool_images/learnlib_architecture.png}
	\caption{Architecture of Learnlib}
	\label{fig:learnlib_arch}
\end{figure}

\paragraph{Learnlib} A library that provides learning algorithms and
infrastructure for automata learning. The learning algorithms consist of a "base
algorithm" whereby the counterexample analysis can be exchanged with other
methods. All the different base algorithms with the examples of variants that
are officially supported are listed below:

\begin{itemize}
	\item L* (base) (Explained in section \ref{sec:fundamental-theory})
	\begin{itemize}
		\item Maler \& Pnueli's \cite{Maler1995}
		\item Rivest \& Schapnire's (Summarized in
		%\ref{sec:rivest-schap-count))
		\item Shahbaz's \cite{Shahbaz2009}
		\item Suffix1by1 \cite{irfan2010}
	\end{itemize}
	\item Obversation Pack (base) \cite{howar2012a}
	\item Kearns \& Vazirani's (base) (Summarized in
	%\ref{sec:classification-trees)
	\item DHC (base) \cite{Merten2012}
	\item TTT (base) %\todo{Reference to theory does not exists, yet}
	\item NL* (base) \cite{Bollig2009}
\end{itemize}

All the algorithms come with both DFA and Mealy versions, expect for DHC and
NL*.

For finding the counterexamples, Learnlib uses Automatalib as well as other
methods like randomized tests. More methods can be found in \cite{Isberner2015}.

Learnlib also offers filters for reducing the amount of queries such as
elimination of duplicate queries. %Is hier een theoriestuk over? It also
contains a parallellization component that can speed up the process by using
multiple teachers and parallel execution of membership
queries\cite{Henrix15}\cite{Howar2012}.

\subsection{Tomte}
\label{ssec:tomte}
Tomte \footnote{In active development and available from
\url{http://tomte.cs.ru.nl/Tomte-0-4}} is a tool that automatically makes
abstractions for automata learning. Essentially, it a connector between the
system under learning(SUL) and the learner. This makes using Learnlib and Libalf
easier, since the user doesn't have to make the mapping.

\begin{figure}[!ht]
	\includegraphics[width=\textwidth]{Tool_images/tomte_network.png}
	\caption{Architecture of Tomte}
	\label{fig:tomte_arch_interactoin}
\end{figure}

The Abstractor, Lookahead Oracle and the Determinizer together form Tomte. The
other two parts are not part of Tomte but it comes with a supplied library
(Learnlib) for making the learner. The makers also have a tool \footnote{SUL
Tool available from http://tomte.cs.ru.nl/Sut-0-4/Description} available for
creating the SUL since they must be modeled after a register automata.

\paragraph{Determinizer} The Determinizer elimates the nondeterministic
behaviour caused by the SUL. Since tools like Learnlib can only analyse
deterministic behaviour, it needs to converted. The theory behind it, is
explained in \cite{Aarts2015}.

\paragraph{Lookahead Oracle} This oracle is used to annotate events of the SUL
with information about the impact on the future behaviour of the SUL. This way,
the Looahead Oracle act as a cache for the Abstractor. The theory and
implementation of this oracle is found in \cite{Aarts2014} and \cite{tomte14}.

\paragraph{Abstractor} The Abstractor is the component that creates the mapping
between the SUL and the learner. The idea behind the Abstractor is to make an
abstraction of the parameter values of the SUL but leaving the input/output
symbols unchanged. It uses counterexample-guided abstraction
refinement\cite{tomte14} for extension of the mapping. In order to make it
scalable, this component also tries to reduce the length of the counterexample
by removing loops and single transitions \cite{Koopman2014}. The complete theory
is found in \cite{tomte14}.

\paragraph{Interaction between the modules} The exchange of messages between
all the modules is as follows(see numbering in figure
\ref{fig:tomte_arch_interactoin}.

\begin{enumerate}
	\item Learner sends an abstract output query to the Abstractor.
	\item Abstracter receives an abstract query, selects a concrete input
	symbol $i$ and send it as an output query to the Lookahead Oracle. If no
	input symbol $i$ exists, return $\perp$ to the Learner.
	\item Lookahead Oracle checks if $i$ is cached. If not, then $s$ is send
	to the Determinizer. Otherwise, go to step 7.
	\item Determinizer transforms the input back to the original behavior of
	the Teacher
	\item Determinizer receives input from the Teacher and transforms this
	nondeterministic behavior.
	\item Determinizer sends concrete symbol $o$ to the Lookahead Oracle which
	caches input output pair $\{i,o\}$
	\item Lookahead oracle sends $o$ together with the annotated information
	that came after the sequence of queries between the last reset query and
	$i$.
	\item Abstractor receives a concrete answer $o$ and the annotated
	information. It uses the annotated information to determine updates to the
	state variables of the Learner. It sends those updates and other
	information to the Learner as an abstract answer.
\end{enumerate}

\end{document}

%%% Local Variables:
%%% mode: latex
%%% TeX-master: "main"
%%% End:
